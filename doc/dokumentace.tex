\documentclass[11pt,a4paper]{scrartcl}
\usepackage[czech]{babel}
\usepackage[utf8]{inputenc}
\usepackage{graphicx}
\usepackage{epstopdf}
\usepackage{float}
\usepackage{amsmath}
\usepackage{listings}
\graphicspath{{.}}

\begin{document}
	\title{Semestrální práce z předmětu KIV/UPS}
	\subtitle{Server - klient : hra Senet}
	\author{Zdeněk Valeš - A13B0458P - valesz@students.zcu.cz}
	\date{13.12. 2016}
	\maketitle
	\newpage
	
	\section{Zadání}
	Naprogramuje síťovou verzi hry Senet. Server naprogramuje v jazyce C, klient může být naprogramován v Jave.
	
	Obojí musí být přeložitelné a spustitelné na školních počítačích, za pomocí automatizačních nástrojů (make, scons, ant, maven).
	
	\section{Popis hry Senet}
	Senet je staroegyptská desková hra pro dva hráče. Každý hráč má 5 kamenů, které jsou na přeskáčku rozestavěny na hrací desce (1. hráč má kámen na poli 1). Hází se dřívky, možné hodnoty jsou 1 - 5 (na rozdíl od kostky mají různou pravděpodobnost). 
	
	Hráč si během tahu vybere jeden kámen, kterým se posune o hozenou hodnotu. Může se pohnout dopředu i do zadu, případně může tah přeskočit. Pokud je na poli, kam se chce hráč pohnout kámen druhého hráče kameny se vymění. Pokud má druhý hráč za sebou dva a více kamenů, výměna není možná. Cílem hry je dostat všechny kameny z hrací desky.
	
	\section{Popis řešení}
	
	\subsection{Protokol}
	Protokol je textově-binární a každá zpráva se skládá ze dvou částí: typ zprávy a obsah zprávy. Typ zprávy je řetězec o právě třech znacích, obsah zprávy má různou délku.
	
	Jednotlivé zprávy jsou uvedeny v následující tabulce:
	
	\begin{tabular} {|c | p{5cm} | c | p{5cm} |}
		\hline
		Název & Popis & Typ & Obsah \\
		\hline
		\hline
		Error & Chybová zpráva. Používá klient i server. & ERR & Dva znaky s kódem chyby.\\
		\hline
		Start game & Zpráva oznamující klientovi začátek hry. Posílá pouze server.& INF & Řetězec 'START\_GAME\verb|nick1|,\verb|nick2|;'.\\
		\hline
		End game & & INF & Řetězec 'END\_GAME\verb|nick|;'. Kde \verb|nick| je nick vítěze. \\
		\hline
		Is alive & & INF &  Řetězec 'ALIVE'.\\
		\hline
		Start turn & & CMD &  Tahová slova obou hráčů. 1. je tahové slovo 1. hráče.\\
		\hline
		OK info & & INF &  Řetězec 'OK'.\\
		\hline
		New player & & CMD  & Řetězec ve tvaru '\verb|délka|\verb|nick|. Kde \verb|délka| je jeden znak (cifra), který určuje délku nicku.\\	
		\hline
		Exit & & INF &  Řetězec 'EXIT'.\\
		\hline
		End turn & & INF  & Tahová slova obou hráčů. 1. je tahové slovo 1. hráče.\\
		\hline
		
		\hline
	\end{tabular}

	\subsection{Tahové slovo}
	Tahové slovo uchovává informaci o tahu. Jedná se o pole 5 čísel, kde každé představuje současnou pozici hráčova kamene. V protokolu je tahové slovo implementováno polem 10 znaků, kde každé dva znaky představují číslo v dvojciferné podobně (1 je tedy '0','1').
		
	\subsection{Server}
	
	\subsection{Klient}
	
	\section{Postup na sestavení a spuštění}
	\subsection{Server}
	Pro úspěšný překlad serveru je nutná knihovna \verb|pthread.h|. Překlad a sestavení lze provést pomocí příkazů \verb|scons|, nebo \verb|cmake| v adresáři \verb|code/server|. Spustitelný soubor \verb|server| je v adresáři \verb|code/server/build|.
	\subsection{Klient}
	Klient lze přeložit a vyexportovat do spustitelného jar pomocí příkazu:
	\begin{lstlisting}
		mvn clean compile assembly:single
	\end{lstlisting}
	v adresáři \verb|code/client|. Vyexportovaný jar pak lze spustit příkazem
	\begin{lstlisting}
		java -jar target/*.jar
	\end{lstlisting} v adresáři \verb|code/client|. Seznam závislostí se náchází v souboru \verb|code/client/pom.ml|.
		
	\section{Závěr}
	
	
\end{document}