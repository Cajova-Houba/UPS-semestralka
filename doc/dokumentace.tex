\documentclass[11pt,a4paper]{scrartcl}
\usepackage[czech]{babel}
\usepackage[utf8]{inputenc}
\usepackage{graphicx}
\usepackage{epstopdf}
\usepackage{float}
\usepackage{amsmath}
\usepackage{listings}
\graphicspath{{.}}

\begin{document}
	\title{Semestrální práce z předmětu KIV/UPS}
	\subtitle{Server - klient : hra Senet}
	\author{Zdeněk Valeš - A13B0458P - valesz@students.zcu.cz}
	\date{13.12. 2016}
	\maketitle
	\newpage
	
	\section{Zadání}
	Naprogramuje síťovou verzi hry Senet. Server naprogramuje v jazyce C, klient může být naprogramován v Jave.
	
	Obojí musí být přeložitelné a spustitelné na školních počítačích, za pomocí automatizačních nástrojů (make, scons, ant, maven).
	
	\section{Popis hry Senet}
	Senet je staroegyptská desková hra pro dva hráče. Každý hráč má 5 kamenů, které jsou na přeskáčku rozestavěny na hrací desce (1. hráč má kámen na poli 1). Hází se dřívky, možné hodnoty jsou 1 - 5 (na rozdíl od kostky mají různou pravděpodobnost). 
	
	Hráč si během tahu vybere jeden kámen, kterým se posune o hozenou hodnotu. Může se pohnout dopředu i do zadu, případně může tah přeskočit. Pokud je na poli, kam se chce hráč pohnout kámen druhého hráče kameny se vymění. Pokud má druhý hráč za sebou dva a více kamenů, výměna není možná. Cílem hry je dostat všechny kameny z hrací desky.
	
	\section{Popis řešení}
	
	\subsection{Protokol}
	
	\subsection{Server}
	
	\subsection{Klient}
	
	\section{Postup na sestavení a spuštění}
	
	\section{Závěr}
	
	
\end{document}